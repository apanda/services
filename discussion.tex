\section{Discussion}
\label{sec:discussion}

Before EC2, a company could only offer a global service once it had learned how to scale its infrastructure (\eg run a datacenter, etc.) and sufficiently invested in the infrastructure. After EC2, companies could focus on meeting customer needs and let Amazon worry about scaling the infrastructure. This has had made it far easier to bring new service ideas to large markets.  Network operators are in a similar position to where Amazon was when first building EC2: the demands of NFV have meant that they are deploying general purpose compute servers at the network edge. These servers are meant to help ISPs rapidly deploy new features, but are necessarily underutilized at most times. \name, similar to many other recent proposals, is therefore inspired by the presence of this spare capacity, but aims to allow application developers, not just ISPs make use of this capacity.

We hope that, similar to EC2, \name will enable application developers to build and deploy network supported services, without worrying about scaling or building out network infrastructure. Right now there are several companies that have significant deployments at the edge (\eg Google, Akamai, Netflix); competing with them would require a new entrant to make a similar infrastructure investment, precluding all but extremely well-funded and targeted efforts from mounting a challenge. With \name, one could deploy a wide variety of edge-based services with very low barriers-to-entry, and we think this might facilitate more rapid innovation in this space.

While one can debate whether \name will have impact, it is clearly feasible for carriers to build and deploy. The Coordinator and Discovery components are straightforward, and carriers can easily deploy racks at the edge to support the required edge services. In this sense, \name's lack of technical depth is a feature, not a bug. The point of our position paper is defining and supporting open network interfaces for application support is trivially within our reach.

The most challenging open question that remains is this: do carriers compete or collaborate in offering \name? If they compete, then each carrier offers a \name-like interface, and individual third-parties can decide how many they need to sign up with to provide adequate edge deployment. Competition might lead to faster adoption of \name-like interfaces, as carriers seek to beat their competitors to market. 

If they collaborate, then \name's interface becomes a standard and network interface generalizes from simple packet delivery to a more service platform.  This would represent the next step in the evolution of the Internet, in which edge services become a fully integrated aspect of carrier networks. This would require solving the question of how carriers peer at the \name level, so that deployment happens at all edges, regardless of the carriers (\ie, if a tenant has signed up for service with one carrier, how does that carrier arrange for other carriers to support that tenant). This is less a technical question than an economic one; technically, it is trivial to disseminate the instantiation information across carriers, but economically it may be hard to agree on the compensation for such peering.

Regardless of how the competition/collaboration is resolved, we believe that incorporating service support is an opportunity whose time has come. The relevant technology (particularly given the advances in SDN and NFV) is readily available, and the overall architecture is conceptually simple and straightforward to build.